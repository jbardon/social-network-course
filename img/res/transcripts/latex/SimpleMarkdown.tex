\documentclass{article}

\usepackage[utf8]{inputenc}

\title{Simple Markdown transcript}
\author{Jérémy Bardon}
\date{}

\begin{document}
\maketitle

Hi, here is the first lesson on Markdown which is about the basics like how to make titles, simple text formatting and lists.
To begin, let's find our project called "social-network-course" which is listed in my GitHub profile
\\\\
During the creation of the project we checked the box to have a simple file automatically generated.
It is called README.md and it's the default file which is displayed when you come on your project page.
\\\\
Now, let's add a title to the document by adding an hash sign before the title.
To make a bold text you can wrap it with two stars or two underlines. An italic style can be done with
only one star. Note that you can make a text bold and italic with three stars.
\\\\
Go back on titles, here we write all the titles for the document
and two of them will be on a lower level.
\\\\
Titles have levels, the first is the bigger and we already used it by adding one hash sign before.
The number of hash signs before is equal to the title level, so if you want a third level title, just set
3 hash signs before your title.
\\\\
If you already type a lot of text you may remark that the text is not on the next line even if you pressed the enter key.
Markdown doesn't take it into account because you have to set two spaces before the line return.
\\\\
The last part of this video is on how to make lists.
First a list with one level: to do this just add one star symbol before each items of your list.
Note that you can also use the minus symbol.
\\\\
To make more complex lists with sub level items, know that the number of tabulations
before an item is equal to the item level in the list
So for one sub list, add one tabulation before each items.
\\\\
If you put in the text a link (so not only in a list) it will automatically display your text 
as a link which will be clickable. And if you click on the link, you will be redirected to the website.

\end{document}