\documentclass{article}

\usepackage[utf8]{inputenc}

\title{Introduction transcript}
\author{Jérémy Bardon}
\date{}

\begin{document}
\maketitle

Hi, welcome in this MOOC called "Collaborative note-taking".
In this course we will use two tools you may already know : Markdown and GitHub.
This MOOC is presented by Philippe Padioleau, Anthony Pena, Brice Thomas and myself 
(Jérémy Bardon)
\\\\
So what do you need when you want to take notes during courses or other situations?
The answer is quite simple: one pen and a paper but you also need to take them fast and simply.
You may need to include some elements like images, diagrams, tables and it can be 
difficult with a paper.
Even more, maybe you want to share your notes with others who can be interested or 
who may want to correct or add some content.
\\\\
The first technology we will use in this course is a language called Markdown.
This language is used to do automatic text formatting.
So for example, if you want to write a title, you don't have to describe the 
style for it, just add an hash sign before your title and it will be automatically formatted as a title.
Markdown also offers a lot of formatting including lists, tables, links and images.
\\\\
The second technology used in this course is a website called GitHub.
This website uses a tool called git which allows you to manage your documents.
All your projects will be hosted on the website so they will be available from everywhere.
The biggest strength of GitHub is that it can manage your documents history.
It means you can visualize your document at particular times. So if you removed a section 
in your document two days ago you can't get it back now. However, with GitHub you can see how your document was two days ago, it's possible.
In addition, you can easily share your documents and allow other to edit them.
\\\\
To conclude, in this course you will learn how to take notes with the Markdown language and use GitHub to manage the different version of your notes.
\end{document}