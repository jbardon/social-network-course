\documentclass{article}

\usepackage[utf8]{inputenc}

\title{Versionning}
\author{Brice Thomas}
\date{}

\begin{document}
\maketitle

Let's talk about versionning. In GitHub, a version is called a \emph{commit}. Here we can review all the different versions of your project. Let's take this one. This one was made 5 days ago. And you can see all the changes that were made during this version. You can see it in a better view. And also in a raw view.
\\\\
Let's go back now and take another version. This one, which was made 3 hours ago. You can see that the deleted lines are in red and the added lines are in green. You can also browse files of a specific version to see how the project looks like at a certain point. And to go back to the lastest version, simply click on the project name.
\\\\
Now, we will make some changes on the \emph{README}. We will separate some parts into paragraphs. This one... and also this one. To do that, let's edit the file. And then it's time to make the changes. Then we put a name on the version, let's call it "separate parts in paragraphs" for example. And when we are done, we commit the changes to make a new version. And if you go back on the main project page, you can see that we are now in the eleventh version. And by clicking on it, we can see all the changes that were made previously.
\\\\
In GitHub you can also see some useful statistics about the project, especially on contributors.
\end{document}
