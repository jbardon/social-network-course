\documentclass{article}

\usepackage[utf8]{inputenc}

\title{More Markdown transcript}
\author{Anthony Pena}
\date{}

\begin{document}
\maketitle

In some situations, you may need to insert quotes in your document.
For doing it, type your quote like a simple text, then just add "$\rangle$" (angle bracket) at the beginning of each lines in your quote and these will be stylish as a quote.
\\\\
As we can see in the Simple Markdown part of this MOOC, links are automatically clickable, but you can replace them with a link text.
It's simple, just surround adress link with parentheses, then add you link text before the first parentheses and surround it by square brackets 
\\\\
It is also possible to insert a table.
To create a table, you just have to type content line by line and separate columns with pipes.
Note you must indicate table head with a line of minus under the first line.
\\\\
There 2 methods to insert an image in a document.
\\\\
The first one is used when you want to add an image which is available online. 
To insert an image get its link.
Use the same syntax as regular link but add "!" (exclamation mark) before the label.
Note if the link is not valid, it will display the specified text.
\\\\
Adding an image from your computer takes more time.
We have to use the issue system.
So create a new issue.
Set a name for it.
Drag'n'drop or select your image.
Copy the link to the image and add it to your document like before.

\end{document}
